\chapter{Códigos}

Há duas formas para inserir um código, em qualquer uma delas é necesário fornecer a opção ``codigo''\ para a classe Estilo, assim ficam disponíveis as ferramentas do pacote \texttt{listings}, as duas principais são exemplificadas abaixo, mas a leitura do manual ou uma busca na internet lhe mostraram
muitas possibilidades.

\section{Inserindo o código de um arquivo}

Essa possibilidade não admite os acentos ortográficos da língua portuguesa, mesmo que estejam em comentário. Mas se o código for copiado para o arquivo tex 
os acentos não serão um problema. 

Para que o \LaTeX\ dê o devido tratamento ao conteúdo do arquivo deve-se informar a qual linguagem o código do arquivo pertence, no exemplo a seguir foi utilizado um código do matlab, por isso foi inserida a chave
$language=Matlab$ para dizer ao \LaTeX\ que o conteúdo do arquivo Laplace.m é um código do MatLab. São aceitas muitas linguagens e alguns dialetos de linguagens, para ver a lista completa veja o manual do pacote
\textsf{listings}, algumas exemplos:

Para inserir código do Matlab use $language=Matlab$

Para inserir código do Maple use $language=Maple$.

Para inserir código do C use $language=C$.

Para inserir código do C++ use $language=C++$.

Para inserir código do Java use $language=Java$.

Para inserir código do Fortran use $language=Fortran$.

Para inserir código do Python use $language=Python$.

Para inserir código do Pascal use $language=Pascal$.

Para inserir código do Octave use $language=Octave$.
 
\begin{tcolorbox}[breakable]
\begin{lstlisting}
\lstinputlisting[language=Matlab,captionpos=b]{Laplace.m}
\end{lstlisting}
\tcblower
\lstinputlisting[language=Matlab,captionpos=b]{Laplace.m}
\end{tcolorbox}

\section{Inserindo o código diretamente}

Para inserir um código basta colocá-lo dentro do ambiente \textsf{lstlisting}. Veja sua sintaxe no manual da classe estilo

\begin{tcolorbox}[breakable,title={Código posto dentro de um ambiente lstlisting}]
\begin{lstlisting}[language=Matlab,
caption={Exemplo de inclusão de código},captionpos=b]
m = input('Digite o numero de linhas ');
n = input('Digite o numero de colunas ');
%%% Montando a Matriz do sistema Bloco diagonal
A = zeros(n,m);
nl1 = n - 1;   np1 = n + 1;  
msize = n*m;   msln = msize - n;
%%% As três diagonais: Diagonal principal, acima e abaixo
A(1,1) = -4.0; %%% Primeiro valor da diagonal principal
for i = 2:msize
   A(i-1,i) = 1.0;  %%% Abaixo
   A(i,i)   =-4.0;  %%% Valores da diagonal principal
   A(i,i-1) = 1.0;  %%% Acima
end
% Correção: Substituindo alguns valores 1 por 0
for i = n:n:msln
   A(i,i+1) = 0.0;
   A(i+1,i) = 0.0;
end
%%% Diagonais afastadas
for i = np1:msize
   A(i,i-n) = 1.0;
   A(i-n,i) = 1.0;
end
%%% Lado direito do sistema (inclui os valores de fronteira)
b=zeros(msize,1);
%%% Atualiza os valores não nulos
for i = n:n:msize
   b(i,1) = -100;
end
u = A\b % Solução do sistema linear.
% Ordenamento matricial dos valores da temperatura
k = 0;
for i = 1:m
   for j = 1:n
      k = k + 1;
      Temp(i,j) = u(k);
   end
end
Temp
\end{lstlisting}
\end{tcolorbox}




