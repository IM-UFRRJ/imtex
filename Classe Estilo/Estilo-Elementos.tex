
\chapter{Classe Estilo: aquisição e uso}


\section{Aquisição}

Para obter a classe \estilo\ envie e-mail para

\begin{center}
	\ttt{\textcolor{estilo}{matematica.ufrrjim@gmail.com}}
\end{center}
e a solicite. Você receberá um e-mail resposta com o arquivo
Estilo.rar\footnote{Esse é um arquivo compactado pelo Winrar,
deve ser aberto pelo Winrar, winzip ou equivalentes, como
o brazip por exemplo.} anexo, descompacte-o em  qualquer
pasta.

Ao descompactar o \ttt{Estilo.rar} você encontrará o manual (pdf) da 
classe estilo e a pasta Monografia, essa pasta contém:
\begin{description}
    \item[Imagens:] pasta contendo imagens utilizadas como exemplo;
    \item[Estilo.cls:] arquivo com todas as definições da classe \estilo;
    \item[logotipo.tex:] utilizado para inserir a logomarca da instituição;
    \item[Logo.png:] logomarca da UFRRJ em \ttt{.png};
    \item[Bibliografia:] Exemplo de banco bibliográfico segundo as regras
	    do Bib\TeX.
    \item[MonoExemplo:] um exemplo de monografia.
\end{description}


% % % \section{Instalação}
% % % 
% % % A classe \estilo\ não exige instalação para ser utilizada,
% % % basta colocar seu arquivo \ttt{.tex} na mesma pasta do
% % % arquivo \estiloc. 
% % %
% % % Contudo, uma instalação permite que a
% % % classe seja utilizada sem que seu arquivo .tex esteja na mesma
% % % pasta do arquivo \estiloc.
% % % 
% % % A instalação consiste em fazer a classe visível ao sistema
% % % TeX/\LaTeX, para isso abra uma pasta(com qualquer nome) no
% % % diretório de instalação do seu sistema \TeX/\LaTeX, ponha o
% % % arquivo \estiloc\ dentro dessa pasta e atualize o FNDB de
% % % sua distribuição, simples assim. Se não sabe o que é FNDB
% % % não tente instalar a classe estilo, use-a sem instalação e
% % % esteja seguro de que não vai correr o risco de bagunçar o
% % % seu sistema.

\section{Utilização}

A classe \estilo\ não exige instalação para ser utilizada,
basta colocar seu arquivo \ttt{.tex} na mesma pasta do
arquivo \estiloc. 

A estilo foi concebida para atender satisfatoriamente o usuário com pouca ou 
nenhuma experiência em \LaTeX. Para utilizá-la defina a primeira linha de 
seu arquivo \ttt{.tex} da seguinte forma

\begin{center}
   \cmc{documentclass}{Estilo}
\end{center}

Assim como as classes tradicionais, a \estilo\ admite parâmetros opcionais 
com os quais é possível alterar e/ou adicionar características e 
funcionalidades. Esses parâmetros são as opções de classe e estão 
descritas na seção \ref{opce}. Sua sintaxe é:

\begin{center}
	\cmco{documentclass}{opção 1, opção 2,$\ldots$}{Estilo}
\end{center}

Para compilar use os mesmos métodos utilizados com as classes nativas do \LaTeX.


\chapter{Características da classe Estilo}


\section{Espaço entre linhas}\label{entrelinha}

Na classe \estilo\, o controle do espaço entre linhas é feito pelo pacote \paco{setspace} e o espaçamento padrão é um e meio, dado pela opção \ttt{onehalfspacing}.

\section{Margens e tipo de papel}\label{margens}

O controle das margens é feito com o pacote \paco{geometry}. O papel padrão na classe \estilo\ é o A$4$, definido por meio da chave a4paper. As margens e o tipo de papal são definidas pelo comando
\begin{center}
	\cmc{geometry}{a4paper,\epar{inner}{4cm},\epar{outer}{3cm},
	     \epar{top}{3cm},\epar{bottom}{2cm}}
\end{center}
Isto quer dizer que
\begin{multicols}{2}
\begin{itemize}
	\item Margem esquerda $=4$ cm
	\item Margem direita $=3$ cm
	\item Margem superior $=3$ cm
	\item Margem inferior $=2$ cm
\end{itemize}
\end{multicols}

\section{Pacotes sempre carregados}

Algumas pacotes sempre são carregados pela classe \estilo, outros ficam 
acessíveis apenas quando uma opção de classe é fornecida. Os pacotes 
sempre carregados são os seguintes:
\begin{description}
   \item[\upacoo{utf8}{inputenc}:] codificação unicode, permite inserir os 
   caracteres acentuados e ç diretamente pelo teclado.
   \item[\upaco{geometry}:] ver~\ref{margens}.
   \item[\upacoo{TS1,T1}{fontenc}:] condificação da fonte de saída, a codificação padrão é T1.
   \item[\upacoo{brazil}{babel}:] traduz para português muitos termos
        produzidos pelo \LaTeX.
   \item[\upaco{scrhack}:] o KOMA usa seu próprio algoritmo para criar
        ambientes flutuantes fornecidos pelo tocbasic. Portanto, pacotes
        como float ou listings usam uma versão antiga e produzem avisos.
        No entanto, Markus Kohm escreveu um pequeno pacote chamado scrhack
        que corrige esse problema. Como os pacotes float e listings são carregados
        pelas opções imagem (padrão) e codigo, foi necessário carregar esse pacote
        na \estilo.
   \item[\upacoo{onehalfspacing}{setspace}:] ver~\ref{entrelinha}.
   \item[\upaco{fix-cm,etex,xspace e calc}:] pacotes técnicos e utilitários carregados sem
   	    qualquer opção.
   \item[\upaco{xcolor}:] suporte a cor. É carregado com as opções
        svgnames, table, x11names, dvipsnames e hyperref.
   \item[\upaco{microtype}:] melhorias para as fontes. É carregado com as opções
        final,kerning,babel,protrusion=true,expansion=true e tracking=true
   \item[\upaco{hyperref}:] utilitário, ver~\ref{links}.
   \item[\upaco{bookmark}:] utilitário
   \item[\upacoo{figure}{hypcap}:]  utilitário
   \item[\upaco{caption}:] formata a legenda de figuras e tabelas criadas
        dentro dos ambientes figure e table. É carregado com as opções
        labelfont=$\{$small,bf,s$\}$, font=$\{$small,sf$\}$, hypcap=false.
   \item[]\cmc{captionsetup}{\epar{justification}{centering}}: legendas de figuras e
        tabelas centralizadas
   \item[\upaco{subcaption}:] ajuda na formatação das legendas.
   \item[\upaco{textcomp}:] fonte, ajustes e símbolos com o encode TS1.
   \item[\upaco{cancel}:] simplificação (cancelamento) de expressões.
   \item[\upaco{rotating}:] suporte para girar coisas.
   \item[\upaco{boxedminipage}:] minipáginas com moldura
   \item[\upaco{makeidx}:] para criar índice remissivo
   \item[\upaco{wallpaper}:] imagem em plano de fundo
   \item[\upaco{multicol}:] conteúdo em mais de uma coluna
   \item[\upaco{amsmath,wasysym,amsfonts,amssymb,amstext}]
   \item[\upaco{amsthm,fixmath}:] demandas variadas em matemática
   \item[\upaco{thmtools}:] utilizado para definir os ambientes para teorema,
               definição, exemplo, corolário, e outros.
   \item[\upaco{enumitem}:] utilitário para listas, foi utilizado para definir
               ajustes para os ambientes enumerate, itemize e description.
   \item[\upaco{float,graphicx}:] carregados por meio da opção \op{imagem} que é padrão.
               O float define o posicionar H e o graphicx permite inserir imagens....  
\end{description}


\section{Estrutura do Documento}

O \LaTeX\ e a classe \estilo\ permite dividir o documento em
parte, capítulo, seção, subseção, subsubseção e parágrafo.
\begin{tcolorbox}
\begin{tabular}{lcl}
   Parte       &$\longrightarrow$& \cmc{part}{título da parte} \\
   Capítulo    &$\longrightarrow$& \cmc{chapter}{título do capítulo} \\
   Seção       &$\longrightarrow$& \cmc{section}{título da seção} \\
   Subseção    &$\longrightarrow$& \cmc{subsection}{título da subseção} \\
   Subsubseção &$\longrightarrow$& \cmc{subsubsection}{título da subsubseção}\\
   Parágrafo   &$\longrightarrow$& \cmc{paragraph}{título do parágrafo}
  \end{tabular}
\end{tcolorbox}

\begin{tcolorbox}[title={Classe estilo: estrutura de um documento científico}]
\begin{lstlisting}
\documentclass{Estilo}

\autor{nome do autor}
\title{título do trabalho}

\begin{document}
   \frontmatter
      Parte pré-textual
   \introducao % fim da parte pré-textual. Inclui \mainmatter

      Conteúdo da introdução

   \ajustes % fim da introdução

      Parte textual

   \bibliographystyle{um estilo de bibliografia}
   \bibliography{nome do arquivo}
\end{document}
\end{lstlisting}
\end{tcolorbox}

\section{Opções da classe Estilo}\label{opce}

A implementação preservou todas as \op{opções} da classe base \ttt{scrbook}.
As opções padrão atendem a todas as normas de formatação de trabalhos científicos do curso de Matemática do Departamento de Tecnologias
e Linguagens - DTL. Além das opções da classe \ttt{scrbook} a classe \estilo\
admite opções próprias, as quais seguem.

\begin{description}
\item[latin:] carrega \upacoo{latin1}{inputenc} e dessa forma ativa suporte à
   codificação iso.
\item[unicode \opp{(padrão)}:] carrega \upacoo{utf8}{inputenc} e dessa forma
   ativa suporte à codificação unicode.
\item[imagem \opp{(padrão)}:] essa opção carrega os pacotes \pacote{float} e
   \pacote{graphicx} e vem ativa por padrão. Podem ser incluídas imagens
   em todos os formatos que o \LaTeX\ aceita, tais como
   \ttt{.png, .pdf, .jpg, .mps (METAPOST)} e outros.
\item[licenciatura \opp{(padrão)}:] ajusta {\sffamily capa} e {\sffamily folha
   de rosto} para uma monografia de graduação relativa a um curso de  licenciatura;
\item[bacharel:] ajusta {\sffamily capa} e {\sffamily folha de rosto}
   para uma monografia de graduação relativa a um curso de bacharelado;
\item[especialista:] ajusta {\sffamily capa} e {\sffamily folha de rosto}
   para uma monografia de curso de especialização;
\item[mestre:] ajusta {\sffamily capa} e {\sffamily folha de rosto} para
   uma dissertação de mestrado;
\item[doutor:] ajusta {\sffamily capa} e {\sffamily folha de rosto} para
   uma tese de doutorado.

\item[Enumeração das páginas:] A enumeração padrão é a da classe
    \textrm{scrbook} e não depende de nenhum pacote. A classe \estilo\
    utiliza o pacote \paco{scrlayer-scrpage} para disponibilizar, na
    forma de opções de classe, cinco tipos de enumeração de páginas, são elas:
    \begin{center}
        \ttt{sofisticada, completa, popular, popularsec, secpopular}
    \end{center}
     Ao utilizar uma dessas opções deve-se
    \begin{enumerate}
    	\item criar a introdução com o comando \com{introducao}.
    	\item finalizar a introdução com o comando \com{ajuste}.
    \end{enumerate}
    Esses comandos ajustam as definições internas e asseguram enumeração correta da parte \emph{textual}.

      \begin{description}
      \item[sofisticada:] todas as páginas são contadas, mas enumeradas a partir na segunda página do sumário com enumeração no cabeçalho, que também recebe uma linha contínua. O rodapé é vazio e a primeira página dos capítulos não recebe numeração. Parte pré-textual enumerada em algarismos romanos as demais em arábicos.
      \begin{itemize}
      	\item se a opção \op{oneside} é carregada(impressão simples, anverso,
      	     apenas frente) o número da página será alinhado à esquerda com o número e título do capítulo alinhados à direita.
      	\item se a opção \op{twoside} é carregada(impressão frente e verso)
      	     o cabeçalho das páginas ímpares terá o número da página alinhado à esquerda com número e título da seção mais próxima alinhados à direita. O cabeçalho das páginas pares terá o número da página alinhado à direita com número e título do capítulo alinhados à esquerda.
      \end{itemize}
      \item[completa:] difere da opção \ttt{sofisticada} somente por enumerar, também, a página inicial dos capítulos e das estruturas definidas a partir do comando \comando{chapter}, como o sumário, lista de tabelas, figuras,$\ldots$;
      \item[popular:] todas as páginas são contadas, mas enumeradas a partir na segunda página do sumário com enumeração no cabeçalho, que também recebe uma linha contínua. O rodapé é vazio e a primeira página dos capítulos não recebe numeração. Parte pré-textual enumerada em algarismos romanos as demais em arábicos.
      \begin{itemize}
      	\item se a opção \op{oneside} é carregada(impressão simples, anverso,
      	apenas frente) o número da página será alinhado à direita com o número e título do capítulo alinhados à esquerda.
      	\item se a opção \op{twoside} é carregada(impressão frente e verso)
      	o cabeçalho das páginas ímpares terá o número da página alinhado à direita com número e título da seção mais próxima alinhados à esquerda. O cabeçalho das páginas pares terá o número da página alinhado à esquerda com número e título do capítulo alinhados à direita.
      \end{itemize}
      \item[popularsec\opp{(padrão)}:] difere da opção \ttt{popular} apenas em documento para impressão simples (opção oneside). A diferença consiste em substituir, no cabeçalho, o título do capítulo pelo título da seção atual;
      \item[secpopular:] difere da opção \ttt{sofisticada} apenas quando a opção \op{oneside} e carregada, ou seja, documento para impressão simples. Nesse caso o número da página é alinha à esquerda e o número e título da seção mais próxima alinhados à direita.
    \end{description}
\item[refkoma\opp{(padrão)}:] referência padrão da classe \ttt{scrbook}.
\item[refnum:] referência numérica.
\item[refaa:] referência autor-ano.
\item[codigo:] carrega o pacote \pacote{listings} e suas ferramentas para 
inserir código, veja os detalhes no capítulo~\ref{codigos}, essa opção é 
desnecessária se for utilizada a opção \op{timesH}. A fonte padrão do código é 
a mesma do texto mas com a família \com{ttfamily}, as opções \op{courier} e 
\op{fontecodigo} permiter alterar e fonte do código.
\item[courier:] carrega \upaco{courier} e deve ser usada preferencialmente em 
conjunto com a \op{timesH} pois altera outras fontes além da fonte dos códigos. 
\item[fontecodigo:] sem carregar \upaco{courier}, altera a fonte do código para 
courier.
\item[palatino:] disponibiliza as seguintes fontes
\begin{itemize}
	\item \upacoo{scaled=.88}{beramono} fonte Bera Monoespaçada.
	\item \upacoo{scaled=.86}{berasans} fonte Bera sem serifa.
	\item \upacoo{sc,osf}{mathpazo} fonte Palatino com small caps e números pequenos.
\end{itemize}
\item[artemisia:] \upaco{gfsartemisia-euler} fonte artemisia.
\item[lmoder:] \upaco{lmodern,libertine} fonte lmodern + libertine (linux)
\item[timesA:] fonte Adobe Times Roman para o corpo do texto e Avantegarde para 
os títulos. Carrega os pacotes \upaco{mathptmx} e \upaco{avant} e utiliza  os 
símbolos matemáticos das fontes Symbol, Chancery e Computer Modern.
\item[timesH:] fonte Times-Helvetica-Courier com os pacotes
\begin{itemize}
	\item \upaco{mathptmx}
	\item \upacoo{scaled=0.92}{helvet}
	\item \upaco{courier}
\end{itemize}
\item[lsec:] essa opção muda o posicionamento dos números das seções e 
subseções. Ative-a, compile e aprecie.
\item[apenas:] permite a criação de apêndices sem anexos.

\item[Página inicial dos capítulos:] a classe estilo disponibiliza cinco opções
    de formatação da página inicial dos capítulos, sumério, referências, anexos e apêndices.
    \begin{description}
    	\item[sonny:] carrega \upacoo{Sonny}{fncychap}
    	\item[lenny:] carrega \upacoo{Lenny}{fncychap}
    	\item[glenn:] carrega \upacoo{Glenn}{fncychap}. Essa opção é incompatível com \epar{bibliography}{totoc} que insere a entrada das referências no sumário, quando carregadas juntas não é criado o retângulo que envolve o título das referências. Assim, quando a opção glenn é carregada a classe estilo não carrega a opção \epar{bibliography}{totoc}, por isso é necessário inserir manualmente a entrada da referências no sumário, o que é feito pondo o comando
    	\begin{tcolorbox}
    	\begin{lstlisting}
    	   \addcontentsline{toc}{chapter}{Referências}
    	\end{lstlisting}
    	\end{tcolorbox}
    	acima do \com{bibliographystyle}. Essa é uma peculiaridade exclusiva da opção glenn.
    	\item[conny:] carrega \upacoo{Conny}{fncychap}
    	\item[rejne:] carrega \upacoo{Rejne}{fncychap}
    	\item[bjornstrup:] carrega \upacoo{Bjornstrup}{fncychap}
    \end{description}
\end{description}

\section{Indentação}

A classe \estilo\ indenta todos os parágrafos,
inclusive o primeiro, seja ele de capítulo, seção ou subseção.

\chapter{Identificação do trabalho}\label{identifica}

Várias informações para identificação do trabalho, composição da
banca examinadora, data, identificação da instituição, cidade e
outras são requeridas para compor a versão final. Na classe \estilo,
para cada uma dessas informações há um comando específico que deve
ficar no preâmbulo. Se um ou mais desses comandos não for declarado
um valor padrão é assumido ou é exibida uma mensagem avisando ao
autor sobre o esquecimento daquela informação.

\section{Instituição}

\subsection{Logomarca}

A classe \estilo\ constrói a capa independente da existência de
uma logomarca. Porém, adicionando os arquivos \ttt{logotipo.tex}
e \ttt{Logo} à mesma pasta de seu arquivo \ttt{.tex} a imagem
contida no arquivo \ttt{Logo} será inserida sobre o nome da instituição.

O arquivo \ttt{Logo} deve está em um dos formatos:
\ttt{.ps, .eps, .png, .pdf, .jpg}. O \ttt{logotipo.tex} é distribuído
com a classe \estilo\ e é o responsável pela inclusão e formatação da imagem.

Se for necessário alterar o tamanho da imagem abra o arquivo
\ttt{logotipo.tex}, o parâmetro \ttt{scale} controla o tamanho da
imagem. Se \epar{scale}{2} ela dobra de tamanho, se \epar{scale}{0.5}
é reduzida à metade.

\subsection{Universidade}

\cmc{universidade}{nome da instituição}: identifica a instituição em que o
trabalho foi desenvolvido. Seu valor padrão é:
\begin{center}
	\ttt{Universidade Federal Rural do Rio de Janeiro}.
\end{center}

\exemplo \cmc{universidade}{Universidade Estadual de Tuntum}

\subsection{Instituto}

\cmc{instituto}{nome do instituto}: identifica o instituto ao qual
pertence o departamento que abriga o curso. Seu valor padrão é:
\begin{center}
	\ttt{Instituto Multidisciplinar}.
\end{center}

\exemplo \cmc{instituto}{Instituto de Ciências Exatas}

\subsection{Departamento}

\cmc{departamento}{nome do departamento}: identifica o departamento que
abriga o curso. Seu valor padrão é:
\begin{center}
	\ttt{Departamento de Tecnologias e Linguagens}.
\end{center}

\exemplo \cmc{departamento}{Departamento de Matemática Aplicada}

\subsection{Curso}

\cmc{curso}{nome do curso}: identifica o curso. Seu valor padrão é: \ttt{Matemática}.

\exemplo \cmc{curso}{Matemática Aplicada e Computacional}

\section{O trabalho}

\subsection{Título do trabalho}

\cmc{title}{título do trabalho}: especifica o título do trabalho.

Esse comando não tem valor padrão, se não for definido será
exibida a mensagem ``\emph{Você não definiu. Defina!}''.

\exemplo \cmc{title}{Zero - O Nada que Existe}

\subsection{Autor}

\cmc{autor}{nome do autor}: define o autor do trabalho. Não tem valor padrão

\exemplo \cmc{autor}{Masha Rostova}

\subsection{Gênero}

\cmc{grauL}{Licenciado ou Licenciada}: para que o autor possa escolher o gênero que deseja que 
conste em sua nmonografia foi definido o comando \com{grauL}. Seu valor padrão é ``\textit{Licenciada}'' por 
que as mulheres são maioria no curso de Matemática. Para alterar seu valor ``\textit{Licenciado}'' acrescente 
\verb|\grauL{Licenciado}| ao preâmbulo de sua monografia. 

\exemplo \cmc{autor}{Masha Rostova}

\grauL{Licenciado}

\subsection{Orientador(a) e Co-orientador(a)}

Existem quatro comandos para definir o orientador(a), são eles
\begin{itemize}
   \item \cmc{orientador}{Aleksandro de Melo}: orientador doutor
   \item \cmc{orientador[m]}{Susan Wouters}: orientadora doutora
   \item \cmc{orientadorm}{Benaia Sobreira de Jesus Lima}: orientador mestre
   \item \cmc{orientadorm[m]}{Katarina Rostova}: orientadora mestre
\end{itemize}
Naturalmente não há valor padrão.


Quando existir co-orientador(a) seu nome será inserido abaixo do nome
do orientador(a). Não havendo, a classe ajusta a distribuição dos
elementos da folha de rosto para evitar que o campo reservado a(o)
co-orientador(a) fique vazio.

Os comandos para co-orientação são obtidos a partir dos comandos
para orientação, simplesmente os precedendo por ``co''. Por exemplo,
quem tem uma co-orientadora mestre o comando é:

\exemplo\cmc{coorientadorm[m]}{Katarina Rostova}

\subsection{Programa}

\cmc{programa}{nome do programa}: define o programa de pós-graduação
ao qual pertence o mestrado/doutorado cursado. Esse comando não possui 
valor padrão e só está definido para as opções de classe mestre e doutor.

\exemplo\cmc{programa}{Matemática Aplicada e Computacional}

\subsection{Diploma}

\cmc{diploma}{nome da área da titulação}: define a área da titulação.
Esse comando não possui valor padrão e só está definido para as opções de classe mestre e doutor.

\exemplo\cmc{diploma}{Biomatemática}

\section{Local e data}

\subsection{Estado}

\cmc{estado}{sigla}: define a sigla do estado da instituição.
Seu valor padrão é: \ttt{RJ}.

\exemplo\cmc{estado}{MA}

\subsection{Cidade}

\cmc{cidade}{nome da cidade}: identifica a cidade onde fica a instituição ou campus em que o autor estudou. Seu valor padrão é: \ttt{Nova Iguaçu}.

\exemplo\cmc{cidade}{Petrópolis}

\subsection{Data}

Para facilitar a manipulação de datas foram definidos os
seguintes comandos:

\begin{description}
 \item[\comando{data}:] Imprime o nome da cidade definida pelo
     comando \coma{cidade}, seguido do dia da compilação(\emph{número}),
     seguido pelo mês(\emph{por extenso}) e ano (\emph{número}) respectivamente;
 \item[\comando{today}(comando do \LaTeX):] Imprime o dia da compilação(\emph{número}),
     seguido pelo mês(\emph{por extenso}) e ano (\emph{número})
     respectivamente;
 \item[\comando{dia}:] Imprime o dia da compilação(\emph{número});
 \item[\comando{mes}:] Imprime o mês da compilação(\emph{número});
 \item[\comando{mesn}:] Imprime o mês da compilação(\emph{nome});
 \item[\comando{ano}:] Imprime o ano da compilação(\emph{número});
\end{description}

\begin{tcolorbox}
\begin{tabular}{ll}
   Comando     & Efeito produzido \\
   \com{data}  & \data            \\
   \com{today} & \today           \\
   \com{dia}   & \dia             \\
   \com{mes}   & \mes             \\
   \com{mesn}  & \mesn            \\
   \com{ano}   & \ano
\end{tabular}
\end{tcolorbox}

Todos esses comandos usam a data da compilação.

\section{Data da defesa}

\cmc{grandedia}{data da defesa}: define o dia da defesa, apresentação pública, do trabalho. Seu valor padrão é \coma{data} que corresponde à data de compilação do documento.

\exemplo \cmc{grandedia}{25 de agosto de 2018}  