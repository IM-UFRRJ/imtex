\chapter{Códigos}\label{codigos}

Há duas formas para inserir um código, ambas exigem que seja carregada a 
opção de classe ``codigo''. Carregada essa opção ficam disponíveis as 
ferramentas do pacote \texttt{listings}, as duas principais são exemplificadas 
a seguir.

\section{Inserindo o código de um arquivo}

Essa possibilidade não admite os acentos ortográficos da língua portuguesa, 
mesmo que estejam em comentário. Mas se o código for copiado para o arquivo tex
os acentos não serão um problema.

Para que o \LaTeX\ dê o devido tratamento ao conteúdo do arquivo 
deve-se informar a qual linguagem o código pertence. São aceitas 
muitas linguagens e alguns dialetos de linguagens, para ver a 
lista completa veja o manual do pacote \pacote{listings}.
Algumas exemplos:

Para inserir código do Matlab use \epar{language}{Matlab}

Para inserir código do Maple use \epar{language}{Maple}.

Para inserir código do C use \epar{language}{C}.

Para inserir código do C++ use \epar{language}{C++}.

Para inserir código do Java use \epar{language}{Java}.

Para inserir código do Fortran use \epar{language}{Fortran}.

Para inserir código do Python use \epar{language}{Python}.

Para inserir código do Pascal use \epar{language}{Pascal}.

Para inserir código do Octave use \epar{language}{Octave}.

No exemplo a seguir foi utilizado um código do matlab, por isso
foi inserida a chave \epar{language}{Matlab} para dizer ao \LaTeX\
que o conteúdo do arquivo Laplace.m é um código do MatLab.

\begin{tcolorbox}[breakable]
\begin{lstlisting}
   \lstinputlisting[language=Matlab,captionpos=b]{Laplace.m}
\end{lstlisting}
\tcblower
\lstinputlisting[language=Matlab,captionpos=b]{Laplace.m}
\end{tcolorbox}

\section{Inserindo o código diretamente}

Para inserir um código basta colocá-lo dentro do ambiente \textsf{lstlisting}. Sua sintaxe é:
\begin{center}
	\begin{minipage}{0.7\textwidth}
		\azul{\coma{begin}}$\{$\verde{\textsf{lstlisting}}$\}$[\op{opções}]
		
		\vspace{0.2cm}
		\hspace{0.5cm}\textit{Conteúdo do ambiente, o código.}
		\vspace{0.2cm}
		
		\azul{\coma{end}}$\{$\textsf{\verde{lstlisting}}$\}$
	\end{minipage}
\end{center}

\begin{tcolorbox}[breakable,title={Código posto dentro de um ambiente lstlisting}]
\begin{lstlisting}[language=Matlab,
caption={Exemplo de inclusão de código},captionpos=b]
m = input('Digite o numero de linhas ');
n = input('Digite o numero de colunas ');
%%% Montando a Matriz do sistema Bloco diagonal
A = zeros(n,m);
nl1 = n - 1;  np1 = n + 1;
msize = n*m;  msln = msize - n;
%%% As três diagonais: Diagonal principal, acima e abaixo
A(1,1) = -4.0; %%% Primeiro valor da diagonal principal
for i = 2:msize
   A(i-1,i) = 1.0;  %%% Abaixo
   A(i,i)   =-4.0;  %%% Valores da diagonal principal
   A(i,i-1) = 1.0;  %%% Acima
end
% Correção: Substituindo alguns valores 1 por 0
for i = n:n:msln
   A(i,i+1) = 0.0;
   A(i+1,i) = 0.0;
end
%%% Diagonais afastadas
for i = np1:msize
   A(i,i-n) = 1.0;
   A(i-n,i) = 1.0;
end
%%% Lado direito do sistema (inclui os valores de fronteira)
b=zeros(msize,1);
%%% Atualiza os valores não nulos
for i = n:n:msize
   b(i,1) = -100;
end
u = A\b % Solução do sistema linear.
% Ordenamento matricial dos valores da temperatura
k = 0;
for i = 1:m
   for j = 1:n
      k = k + 1;
      Temp(i,j) = u(k);
   end
end
Temp
\end{lstlisting}

\end{tcolorbox}




