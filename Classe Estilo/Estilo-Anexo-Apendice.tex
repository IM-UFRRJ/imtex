
\anexo
\chapter{Deus}

\section{Seja grato}
Tu sempre foste meu melhor amigo, Aquele que nunca me abandonou 
ou desamparou; e por tudo isso Te agradeço, meu Deus!


Só Deus é digno de toda a honra, glória e louvor!

Em primeiro lugar, devemos louvar a Deus porque Ele merece. O nosso louvor não pode depender das nossas emoções, porque as nossas emoções mudam, mas Deus é sempre igual.


\chapter{A vida}

A vida gosta de quem gosta dela e quem tem quem lhe chore morre todo dia.

\section{Você aprende}

Aprende que falar pode aliviar dores emocionais, e descobre que se leva anos para se construir confiança e apenas segundos para destruí-la, e que você pode fazer coisas em um instante, das quais se arrependerá pelo resto da vida; aprende que verdadeiras amizades continuam a crescer mesmo a longas distâncias, e o que importa não é o que você tem na vida, mas quem você tem na vida, e que bons amigos são a família que nos permitiram escolher.

Aprende que não temos que mudar de amigos se compreendemos que eles mudam; percebe que seu melhor amigo e você podem fazer qualquer coisa, ou nada, e terem bons momentos juntos.

Descobre que as pessoas com quem você mais se importa na vida são tomadas de você muito depressa, por isso sempre devemos deixar as pessoas que amamos com palavras amorosas; pode ser a última vez que as vejamos.

E descobre que não importa o quanto você se importa, algumas pessoas simplesmente não se importam.

Aprende que não importa em quantos pedaços seu coração foi partido, o mundo não pára para que você o conserte.

Portanto$\ldots$ plante seu jardim e decore sua alma, ao invés de esperar que alguém lhe traga flores.


\apendice
\chapter{Um teste de apêndice}

\section{Diferenças}

\subsection{Anexo}
Seu trabalho usa fortemente resultados pouco conhecidos
ou extensos demais para serem inseridos no corpo do texto. No
final você pode inserir os tais resultados, a coleção deles
constitui-se um ou mais anexos.

\subsection{Apêndice}

Seu trabalho aborda a vida de Charles Chaplin. No final você
pode inserir fotos de cartazes dos filmes dele, algumas de suas 
frases famosas, fotografias dele, citar um poema ou parte dele $\ldots$. 
Como essas informações não são necessárias para o bom entendimento do 
seu trabalho elas constituem um apêndice.


Note a diferença entre anexo e apêndice, o anexo contém informações 
indispensáveis para a compreensão do trabalho enquanto o apêndice 
contém informações acessórias e cuja ausência não comprometem a 
compreensão do texto ou parte dele.

\section{Erro}
pdftex warning (ext4): destination with the same identifier has been already used, duplicate ignored


\chapter{A vida continua}
\section{Você}

Depois de um tempo você aprende que o sol pode queimar se ficarmos expostos a ele durante muito tempo. E aprende que não importa o quanto você se importe: algumas pessoas simplesmente não se importam$\ldots$. E aceita que não importa o quão boa seja uma pessoa, ela vai ferí-lo de vez em quando e, por isto, você precisa estar sempre disposto a perdoá-la.
