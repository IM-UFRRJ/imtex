
\chapter{Elementos textuais}

\section{Citações}

Foi definido o ambiente \ttt{citei} para citações longas com
mais de três ($03$) linhas. Este ambiente atende as exigências das
normas da ABNT.

\begin{tcolorbox}
\begin{lstlisting}
\begin{citei}
          Fragmentos do poema ``Você Aprende''

   Aprende que quando está com raiva tem o direito de estar
   com raiva, mas isso não te dá o direito de ser cruel.
   Aprende que não importa em quantos pedaços seu coração foi
   partido, o mundo não pára para que você o conserte.
   Portanto, plante seu jardim e decore sua alma, ao invés de
   esperar que alguém lhe traga flores.

   \hfill William Shakespeare
\end{citei}
\end{lstlisting}
depois de compilado mostra

\begin{citei}
             Fragmentos do poema ``Você Aprende''

Aprende que quando está com raiva tem o direito de estar
com raiva, mas isso não te dá o direito de ser cruel.
Aprende que não importa em quantos pedaços seu coração foi
partido, o mundo não pára para que você o conserte.
Portanto, plante seu jardim e decore sua alma, ao invés de
esperar que alguém lhe traga flores.

\hfill William Shakespeare
\end{citei}
\end{tcolorbox}


\section{Referência cruzada}

A classe \estilo{} utiliza o mecanismo padrão do \LaTeX\ para construir referências cruzadas.
Na seção~\ref{cruzada} encontra-se descrição e exemplos desse procedimento.

\section{Cor}

A classe \estilo{} aceita todos os pacotes do \LaTeX{} dedicados a
inserção de cor em elementos do texto. Porém, dada sua versatilidade
e potência escolheu-se o pacote \pacote{xcolor}, que é carregado assim
\begin{center}
	\upacoo{svgnames,table,x11names,dvipsnames,hyperref}{xcolor}
\end{center}

\section{Links} \label{links}
O pacote \paco{hyperref} é carregado com as opções
\begin{description}
  \item[colorlinks:] suporte a cor em todos os
     links criados pelo pacote \pacote{hyperref}. Seu valor padrão
     na classe \estilo{} é \hfill \epar{colorlinks}{true}
  \item[citecolor:] define a cor das citações inseridas com o
     comando \comando{cite}. Seu valor padrão
\hfill \epar{citecolor}{Navy}
  \item[urlcolor:] Define a cor dos endereços web (páginas de internet)
     inseridos com os comandos do \paco{hyperref} -\comando{url},
     \comando{href}, $\ldots$. O valor padrão é \hfill \epar{urlcolor}{Navy}
  \item[linkcolor:] Atribui cor aos ítens do sumário, da lista de
     tabelas, da lista de figuras, às referências cruzadas e indicadores
     de nota de rodapé. O valor padrão é \hfill  \epar{linkcolor}{Navy};
  \item[breaklinks:] faz quebra de link de forma inteligente;
  \item[pdfstartview=FitB:] abre o pdf em tamanho normal;
  \item[pdfpagelabels:] define rótulos para páginas em pdf;
  \item[linktocpage=true:] põe link no número da página no sumário, na lista
     de tabelas e na lista de figuras;
  \item[bookmarksopen=true:] bookmark exibido no pdf;
  \item[bookmarksopenlevel=1:] bookmark expandido até o nível $1$;
  \item[bookmarksnumbered=true:] bookmark enumerado;
\end{description}

O \pacote{hyperref} cria {\itshape links} e adiciona cor aos itens do
sumário, da lista de tabelas e da lista de figuras. Também cria
{\itshape links} para as referências cruzadas, notas de rodapé, citações,
e muito mais, tudo isso é feito automaticamente na classe estilo.

Use o comando \com{hypersetup} no preâmbulo de seu documento para
inserir mais opções ao pacote \paco{hyperref}. Se for acrescentado o código
\begin{tcolorbox}
\begin{lstlisting}
   \hypersetup{pdfpagemode=FullScreen} %%% Tela cheia - full-scream.
\end{lstlisting}
\end{tcolorbox}
ao preâmbulo, sempre que o documento for compilado o pdf abrirá em
modo de tela cheia.

\section{Elementos matemáticos}

Ao redigir um trabalho algumas tarefas consagram-se corriqueiras,
neste caso é conveniente definir um comando para executá-las de
forma mais simples e padronizá-las. A classe \estilo\ buscou
atender algumas formas mais básicas desse tipo de demanda
definindo ou redefinindo comandos.

\subsection{Conjuntos Numéricos}

Com mais ou menos intensidade, os conjuntos numéricos estão
presentes em qualquer texto matemático e, não raro, o emprego
de notação é mais apropriado e até imperativo. Para sua
manipulação foram definidos na classe \estilo\ comandos
para inserir a notação dos principais conjuntos numéricos.

Todos os comandos do quadro a seguir exigem o pacote \pacote{amsfonts} da
\emph{American Mathematical Society - AMS}.
\begin{tcolorbox}
	\begin{tabular}{llcl}
	Conjunto dos números Complexos   & \verb|$\C$ ou \C| & $\longrightarrow$ &\C \\
	Conjunto dos números Irracionais & \verb|$\I$ ou \I| & $\longrightarrow$ &\I \\
	Conjunto dos números Naturais    & \verb|$\N$ ou \N| & $\longrightarrow$ &\N \\
	Conjunto dos números Primos      & \verb|$\P$ ou \P| & $\longrightarrow$ &\P \\
	Conjunto dos números Racionais   & \verb|$\Q$ ou \Q| & $\longrightarrow$ &\Q \\
	Conjunto dos números Reais       & \verb|$\R$ ou \R| & $\longrightarrow$ &\R \\
	Conjunto dos números Inteiros    & \verb|$\Z$ ou \Z| & $\longrightarrow$ &\Z
	\end{tabular}
\end{tcolorbox}


\begin{tcolorbox}
	\begin{lstlisting}
	Se $x\in\I$ então $x\notin\N,\,\,x\notin\P,\,\,x\notin\Z$
	e $x\notin\Q$, mas $x\in\R$.
	\end{lstlisting}
	\tcblower
	
	Se $x\in\I$ então $x\notin\N,\,\,x\notin\P,\,\,x\notin\Z$ e
	$x\notin\Q$, mas $x\in\R$.
\end{tcolorbox}

\subsection{Funções}

Os comandos seguintes definem notação para algumas funções usuais
\begin{tcolorbox}
  \begin{tabular}{lcl}
    \verb|$\sen{x}$ ou \sen{x}|   & $\longrightarrow$ & \sen{x} \\
    \verb|$\Sen{x}$ ou \Sen{x}|   & $\longrightarrow$ & \Sen{x} \\
    \verb|$\Cos{x}$ ou \Cos{x}|   & $\longrightarrow$ & \Cos{x} \\
    \verb|$\tg{x}$ ou \tg{x}|     & $\longrightarrow$ & \tg{x}  \\
    \verb|$\cotg{x}$ ou \cotg{x}| & $\longrightarrow$ & \cotg{x}  \\
    \verb|$\comp{f}{g}(x)$ |      & $\longrightarrow$ & $\comp{f}{g}(x)$
  \end{tabular}
\end{tcolorbox}

Em alternativa aos comandos do \LaTeX{} \comando{sin}, \comando{cos} e
\comando{tan} foram definidos os comandos \comando{sen}, \comando{Cos}
e \comando{tg}. Eles possuem um parâmetro opcional para construir
potências e seu argumento fica entre parênteses que se ajustam
automaticamente ao tamanho do argumento.
\begin{tcolorbox}
\begin{lstlisting}
   \Sen[op]{arg.}    \Cos[op]{arg.}    \tg[op]{arg.}
\end{lstlisting}
\end{tcolorbox}
Se o parâmetro opcional não for declarado a classe entende que é
vazio, ou seja, sem potência.

\begin{tcolorbox}
\begin{lstlisting}
\[
   \Sen{\dfrac{2\pi x}{5}} + \tg[3]{\dfrac{2\pi x}{5}} +
   \Cos[2]{\dfrac{2\pi x}{5}} \neq{1}
\]
\end{lstlisting}
\tcblower
\[
\Sen{\dfrac{2\pi x}{5}} + \tg[3]{\dfrac{2\pi x}{5}} +
\Cos[2]{\dfrac{2\pi x}{5}} \neq{1}
\]
\end{tcolorbox}

\begin{tcolorbox}	
	\verb|\comp{f}{g}(x)\overset{=} f(g(x))| $\longrightarrow$	
	$\comp{f}{g}(x) \overset{=} f(g(x))$
\end{tcolorbox}

\subsection{Números Naturais}

Os conceitos de Mínimo Múltiplo Comum - mmc e Máximo Divisor
Comum - mdc são utilitários a quem lida com teoria dos números.
A classe \estilo\ oferece um comando para cada um deles:
\begin{tcolorbox}
	\begin{tabular}{lcl}
		\verb|$\mmc{a,b}$ ou \mmc{a,b}|   & $\longrightarrow$ & \mmc{a,b} \\
		\verb|$\mdc{a,b}$ ou \mdc{a,b}|   & $\longrightarrow$ & \mdc{a,b}
	\end{tabular}
\end{tcolorbox}

\subsection{Unidades}

O comando \comando{unidade} usa fonte \ttt{roman} definida pelo \comando{mathrm} e ajusta corretamente o espaço entre o número e a unidade.
Ele funciona dentro e fora do modo matemático.
\begin{tcolorbox}
	\begin{tabular}{lcl}
		\verb|$\unidade{108}{m/s}$ ou \unidade{108}{m/s}| & $\longrightarrow$ & \unidade{108}{m/s} \\
		\verb|$\unidade{3}{N/m^2}$ ou \unidade{3}{N/m^2}| & $\longrightarrow$ & \unidade{3}{N/m^2}
	\end{tabular}
\end{tcolorbox}

\subsection{O grau}
O comando \com{grau} insere a notação do grau (bolinha).
\[
    \verb|$30\grau$|\quad \longrightarrow\quad  30\grau
\]

\subsection{Álgebra linear}

Os comandos \com{vetor} e \com{veto} criam um vetor com $n-$coordenadas.
Eles admitem dois parâmetros, um obrigatório - para atribuir
nome ao vetor (uma letra minúscula) e um opcional - para
especificar o número de coordenadas. O argumento opcional tem valor
padrão $n$.


\begin{tcolorbox}
\begin{lstlisting}
   \vetor[Número de coordenadas]{nome do vetor (letra)} %%% Com seta
   \veto[Número de coordenadas]{nome do vetor (letra)}  %%% Sem seta
\end{lstlisting}
\tcblower
\begin{tabular}{lcl}
   \verb|$\vetor{u}$ ou \vetor{u}|       & $\longrightarrow $& \vetor{u}    \\
   \verb|$\vetor[7]{u}$ ou \vetor[7]{u}| & $\longrightarrow$ & \vetor[7]{u} \\
   \verb|$\veto{u}$ ou \veto{u}|         & $\longrightarrow$ & \veto{u}     \\
   \verb|$\veto[13]{u}$ ou \veto[13]{u}| & $\longrightarrow$ & \veto[13]{u}
\end{tabular}
\end{tcolorbox}

A única diferença entre os comandos \com{vetor} e \com{veto} é a seta sobre a letra que identifica o vetor.

O comando \com{base} nomeia e descreve os vetores de uma base, sua sintaxe é
\begin{center}
\cmcc{base}{indice}{nome da base}{nome dos vetores da base}
\end{center}

\begin{tcolorbox}
\begin{tabular}{lcl}
   \verb|$\base{B}{u}$ ou \base{B}{u}| & $\longrightarrow $&\base{B}{u} \\
   \verb|$\base[r]{B}{v}$ ou \base[r]{B}{v}| & $\longrightarrow$ & \base[r]{B}{v}
\end{tabular}
\end{tcolorbox}
	
O comando \com{inter} produz a notação do produto interno.
\begin{tcolorbox}
    \verb|\inter{x + y,z} = \inter{x,z} + \inter{y,z}|
    \tcblower
    \inter{x + y,z} = \inter{x,z} + \inter{y,z}
\end{tcolorbox}

O comando \com{nor} introduz a notação de norma de vetor. Havendo necessidade, o tipo de norma pode ser especificada utilizando o parâmetro opcional, o qual, por omissão, é vazio.
\begin{tcolorbox}
		\verb|$\nor[\infty]{x} \leq \nor[2]{x} \leq \nor[1]{x}$|  e
		
		\verb|$\nor{x+y} \leq \nor{x} + \nor{y}$|
		\tcblower
		$\nor[\infty]{x} \leq \nor[2]{x} \leq \nor[1]{x}$ \qquad e \qquad		
		$\nor{x+y} \leq \nor{x} + \nor{y}$
		
\end{tcolorbox}

\subsection{Empilhando símbolos}

Os comandos \coma{stackrel} e \com{atop} sempre foram utilizados para empilhar coisas, mas agora
estão obsoletos, os atuais são \com{overset} e \com{underset}, ambos do pacote amsmath, mas esses
comandos tem uma característica que pode não ser apreciada, eles reduzem um dois símbolos que empilham

\begin{tcolorbox}
\begin{lstlisting}
\[
f(x) \overset{\longleftarrow}{\longrightarrow} x\ln(1+x) \e
f(x) \underset{\longleftarrow}{\longrightarrow} x\ln(1+x)
\]
\end{lstlisting}
\tcblower
\[
f(x) \overset{\longleftarrow}{\longrightarrow} x\ln(1+x) \e
f(x) \underset{\longleftarrow}{\longrightarrow} x\ln(1+x)
\]
\end{tcolorbox}
\begin{tcolorbox}[title={Um singelo \coma{hbox} faz muita diferença}]
\begin{lstlisting}
\[
f(x) \overset{\hbox{$\longrightarrow$}}{\longrightarrow} x\ln(1+x) \e
f(x) \underset{\hbox{$\longrightarrow$}}{\longrightarrow} x\ln(1+x)
\]
\end{lstlisting}
\tcblower
\[
f(x) \overset{\hbox{$\longrightarrow$}}{\longrightarrow} x\ln(1+x) \e
f(x) \underset{\hbox{$\longrightarrow$}}{\longrightarrow} x\ln(1+x)
\]
\end{tcolorbox}


\begin{tcolorbox}
\begin{lstlisting}
\[
   f'(x)\overset{\hbox{def.}}{=}\lim_{h \to 0}\dfrac{f(x+h)-f(x)}{h} \e
   \overset{x>0}{\Longrightarrow} \ou A\overset{f}{\longrightarrow}B
\]
\end{lstlisting}
\tcblower
\[
   f'(x) \overset{\hbox{def.}}{=} \lim_{h \to 0} \dfrac{f(x+h) - f(x)}{h} \e
   \overset{x>0}{\Longrightarrow} \ou A\overset{f}{\longrightarrow}B
\]
\end{tcolorbox}



\subsection{Soma com índices e soma com potências}

O comando \com{soma} introduz a notação para uma soma
finita com índices, por padrão, com $n$ termos, mas um
parâmetro opcional permite alterar esse valor.
\begin{tcolorbox}
 \begin{tabular}{lcl}
  \verb|$\soma{x}$ ou \soma{x}|         & $\longrightarrow$ &\soma{x}     \\
  \verb|$\soma[m]{x}$ ou \soma[m]{x}|   & $\longrightarrow$ &\soma[m]{x}  \\
  \verb|$\soma[29]{x}$ ou \soma[29]{x}| & $\longrightarrow$ &\soma[29]{x}
\end{tabular}
\end{tcolorbox}

O comando \com{pot} introduz a notação para uma soma
finita de potências, por padrão, com $n$ termos, mas um
parâmetro opcional permite alterar esse valor.
\begin{tcolorbox}
 \begin{tabular}{lcl}
  \verb|$\pot{x}$ ou \pot{x}|        & $\longrightarrow$ & \pot{x}    \\
  \verb|$\pot[m]{x}$ ou \pot[m]{x}|  & $\longrightarrow$ & \pot[m]{x} \\
  \verb|$\pot[29]{x}$ ou \pot[29]{x}|& $\longrightarrow$ & \pot[29]{x}
\end{tabular}
\end{tcolorbox}

\begin{tcolorbox}
\begin{lstlisting}
\[
   1+\pot{x} = \dfrac{ 1-x^{n+1} }{1-x}
\]
\end{lstlisting}
\tcblower
\[
1+\pot{x} = \dfrac{1-x^{n+1}}{1-x}
\]
\end{tcolorbox}

\section{Ambientes}

A classe estilo utiliza o pacote \pacote{thmtools} para definir os ambientes de manipulação de entes matemáticos. Foi definido ambiente para exemplos, lemas, proposições, teoremas, corolários e definições.

Todos os ambientes admitem um argumento opcional que pode ser utilizado para atribuir um nome ao teorema, 
definição, $\ldots$. Examine os exemplos em cada subseção que segue.

\subsection{Definição}
\begin{tcolorbox}
\begin{lstlisting}
\begin{define}[Funcional sublinear]
   Sejam $E$ um espaço vetorial real. Dizemos que uma função
   $f: E \longrightarrow \R$ é um funcional sublinear se
   \[ f(x + y) \leqslant f(x) + f(y)  \e  f(\alpha x) = \alpha f(x) \]
   para quaisquer $x, y\in E$ e para todo $\alpha\in\R,\,\alpha\geqslant 0$.
\end{define}
\end{lstlisting}
\tcblower
\begin{define}[Funcional sublinear]
	Sejam $E$ um espaço vetorial real. Dizemos que uma função
	$f: E \longrightarrow \R$ é um funcional sublinear se
	\[
	f(x + y) \leqslant f(x) + f(y)  \e  f(\alpha x) = \alpha f(x)
	\]
	para quaisquer $x, y\in E$ e para todo $\alpha\in\R,\,\alpha\geqslant 0$.
\end{define}
\end{tcolorbox}

\subsection{Lema}
\begin{tcolorbox}
\begin{lstlisting}
\begin{lema}[Zorn]
   Seja $A$ um conjunto parcialmente ordenado. Se todo subconjunto
   totalmente ordenado de $A$ possui um limitante superior, então $A$
   tem pelo menos um elemento maximal.
\end{lema}
\end{lstlisting}
	\tcblower
\begin{lema}[Zorn]
	Seja $A$ um conjunto parcialmente ordenado. Se todo subconjunto
	totalmente ordenado de $A$ possui um limitante superior, então $A$
	tem pelo menos um elemento maximal.
\end{lema}
\end{tcolorbox}

\subsection{Proposição}

Analogamente tem-se o ambiente \ttt{prop} para proposição. Ele funciona do 
mesmo modo que o ambiente lema. Veja um exemplo na seção~\ref{prova}.

\subsection{Teorema}
\begin{tcolorbox}[breakable]
\begin{lstlisting}
\begin{teorema}[Forma analítica do teorema de Hahn-Banach]
   Sejam $E$ um espaço vetorial sobre \R, $U\subseteq E$ um subespaço vetorial e $p$ um funcional sublinear em $E$. Para todo funcional linear $f:U\longrightarrow \R$ tal que $f(x)\leqslant p(x)$ para todo $x\in U$, existe uma extensão linear de $f$, ou seja, existe um funcional linear $F:E\longrightarrow \R$ tal que
\[ F(u)=f(u)\,\,\,\forall\,u\in U  \e  F(x)\leqslant p(x)\,\,\,
   \forall\, x\in E.\]
\end{teorema}
\end{lstlisting}
	\tcblower
	\begin{teorema}[Forma analítica do teorema de Hahn-Banach]
		Sejam $E$ um espaço vetorial sobre \R, $U\subseteq E$ um subespaço vetorial e $p$ um funcional sublinear em $E$. Para todo funcional linear $f:U\longrightarrow \R$ tal que $f(x)\leqslant p(x)$ para todo $x\in U$, existe uma extensão linear de $f$, ou seja, existe um funcional linear $F:E\longrightarrow \R$ tal que
		\[
		 F(u)=f(u)\,\,\,\forall\,u\in U  \e  F(x)\leqslant p(x)\,\,\,
		 \forall\, x\in E.
		\]
	\end{teorema}
\end{tcolorbox}

\begin{tcolorbox}
\begin{lstlisting}
\begin{teorema}[Teorema de Hahn-Banach em espaços normados]
   Sejam $E$ um espaço vetorial normado, $U\subseteq E$ um subespaço vetorial e $f:U\longrightarrow \R$ um funcional linear limitado. Então existe uma extensão limitada de $f$ sobre $E$ com a mesma norma de $f$, ou seja, existe um funcional linear $F:E\longrightarrow \R$ tal que
   \[
      F(u)=f(u)\,\,\,\forall\,u\in U  \e  \nor{F}_E=\nor{f}_U.
   \]
\end{teorema}
\end{lstlisting}
	\tcblower
\begin{teorema}[Teorema de Hahn-Banach em espaços normados]
	Sejam $E$ um espaço vetorial normado, $U\subseteq E$ um subespaço vetorial e $f:U\longrightarrow \R$ um funcional linear limitado. Então existe uma extensão limitada de $f$ sobre $E$ com a mesma norma de $f$, ou seja, existe um funcional linear $F:E\longrightarrow \R$ tal que
	\[
	   F(u)=f(u)\,\,\,\forall\,u\in U  \e  \nor[E]{F}=\nor[U]{f}.
	\]
\end{teorema}
\end{tcolorbox}

\subsection{Corolário}
\begin{tcolorbox}
\begin{lstlisting}
\begin{coro}
   Sejam $E$ um espaço vetorial normado e $x_0\in E,\,x_0\neq 0$. Então, existe um funcional linear limitado $f:E\longrightarrow \R$ tal que
   \[
      \nor{f} = 1 \e f(x_0) = \nor{x_0}
   \]
\end{coro}
\end{lstlisting}
\tcblower
\begin{coro}
   Sejam $E$ um espaço vetorial normado. Para todo elemento não nulo $x_0\in E$  existe um funcional linear limitado $f:E\longrightarrow \R$  tal que
   \[
      \nor{f} = 1 \e f(x_0) = \nor{x_0}
   \]
\end{coro}
\end{tcolorbox}

A menos que seja definido pelo usuário, o contador de um ambiente criado com o pacote \pacote{thmtools} recebe o mesmo nome do ambiente. Esse contadores podem ser redefinido com os comandos usais do \LaTeX.

Para alterar a enumerar dos corolários de modo a começar com IV, deve-se redefinir a enumeração para algarismos romanos e 
o contador para 3, como segue

\begin{tcolorbox}
\begin{lstlisting}
\setcounter{coro}{3}   %%% Redefine o contador coro para 3
\renewcommand{\thecoro}{\Roman{coro}}
%%% Altera o contador coro para algarísmos romanos maiúsculos
\begin{coro}
   Sejam $E$ um espaço vetorial normado. Para todo elemento não nulo
   $x_0\in E$  existe um funcional linear limitado $f:E\longrightarrow \R$ tal que
   \[
      \nor{f} = 1 \e f(x_0) = \nor{x_0}
   \]
\end{coro}
\end{lstlisting}
	\tcblower
\setcounter{coro}{3}   %%%Redefine o contador para 3
\renewcommand{\thecoro}{\Roman{coro}}%%% Romanos maiúsculos
\begin{coro}
   Sejam $E$ um espaço vetorial normado. Para todo elemento não nulo
   $x_0\in E$  existe um funcional linear limitado $f:E\longrightarrow \R$ tal que
   \[
      \nor{f} = 1 \e f(x_0) = \nor{x_0}
   \]
\end{coro}
\end{tcolorbox}

\subsection{Prova}

\begin{tcolorbox}
\begin{lstlisting}
\begin{prop}\label{prova}
   No espaço $l^p$ vale a desigualdade triangular.
\end{prop}
\begin{prova}
Sejam $x,y,z\in l^p$ então
\begin{eqnarray*}
	d(x,y)&=&\sqrt[p]{\sum_{j=1}^{\infty}|x_j-y_j|^p}\\
	      &=&\sqrt[p]{\sum_{j=1}^{\infty}|x_j-z_j+z_j-y_j|^p}\\
	      &\leqslant&\sqrt[p]{\sum_{j=1}^{\infty}
	       \left(|x_j-z_j|+|z_j-y_j|\right)^p}\\
	      &\overset{*}{\leqslant}&\sqrt[p]{\sum_{j=1}^{\infty}
	       \left(|x_j-z_j|\right)^p}+\sqrt[p]{\sum_{j=1}^{\infty}
	       \left|z_j-y_j|\right)^p}\\
	      &=& d(x,z) + d(z,y)
\end{eqnarray*}

$*$ Pela desigualdade de Minkowski.
\end{prova}
\end{lstlisting}

\tcblower

\begin{prop}\label{prova}
   No espaço $l^p$ vale a desigualdade triangular.
\end{prop}
\begin{prova}
Sejam $x,y,z\in l^p$ então
\begin{eqnarray*}
	d(x,y)&=&\sqrt[p]{\sum_{j=1}^{\infty}|x_j-y_j|^p}\\
	      &=&\sqrt[p]{\sum_{j=1}^{\infty}|x_j-z_j+z_j-y_j|^p}\\
	      &\leqslant&\sqrt[p]{\sum_{j=1}^{\infty}
	       \left(|x_j-z_j|+|z_j-y_j|\right)^p}\\
	      &\overset{*}{\leqslant}&\sqrt[p]{\sum_{j=1}^{\infty}
	       \left(|x_j-z_j|\right)^p}+\sqrt[p]{\sum_{j=1}^{\infty}
	       \left|z_j-y_j|\right)^p}\\
	      &=& d(x,z) + d(z,y)
\end{eqnarray*}

$*$ Pela desigualdade de Minkowski.
\end{prova}
\end{tcolorbox}

\subsection{Exemplo}
\begin{tcolorbox}
\begin{lstlisting}
\begin{exemplo}[hiperplanos]
   Se $E = \rn[2]$, sabemos que todo funcional linear $f$ definido em \rn[2] pode ser escrito na forma $f(x,y) = ax + by$ para todo $(x,y)\in\rn[2]$, para algum $a,b\in\R$. Se $a^2 + b^2 \neq 0$, os hiperplanos afins de \rn[2] são da forma
   \[
   H =\{(x,y)\in \rn[2] \mid ax + by = \alpha\},
   \]
   ou seja, são as retas no plano. Para $E=\rn[3]$, os hiperplanos são da forma
   \[
      H =\{(x,y,z)\in \rn[2] \mid ax + by + cz= \alpha\},
   \]
   em que $a^2 + b^2 + c^2\neq 0$, que nada mais são do que planos no espaço.
\end{exemplo}
\end{lstlisting}
\tcblower
\begin{exemplo}[hiperplanos]
	Se $E = \rn[2]$, sabemos que todo funcional linear $f$ definido em \rn[2] pode ser escrito na forma $f(x,y) = ax + by$ para todo $(x,y)\in\rn[2]$, para algum $a,b\in\R$. Se $a^2 + b^2 \neq 0$, os hiperplanos afins de \rn[2] são da forma
	\[
	H =\{(x,y)\in \rn[2] \mid ax + by = \alpha\},
	\]
	ou seja, são as retas no plano. Para $E=\rn[3]$, os hiperplanos são da forma
	\[
	H =\{(x,y,z)\in \rn[3] \mid ax + by + cz= \alpha\},
	\]
	em que $a^2 + b^2 + c^2\neq 0$, ou seja, são os planos no espaço.
\end{exemplo}
\end{tcolorbox}

\subsection{Fórmulas enumeradas - ambiente equation}
 Em algumas situações é conveniente e em outras é necessários enumerar um resultado, uma equação, desigualdade ou outros elementos. A classe estilo suporta todas as ferramentas de enumeração
 do \LaTeX, duas delas são exemplificadas a seguir.
\begin{tcolorbox}[title={Ambiente equation}]
 O ambiente \verb|\[   \]| foi muito utilizados neste manual, a diferença dele para o 
 \ttt{equation} é a numeração que o equation insere. 
\begin{lstlisting}
\[  H =\{(x,y)\in \rn[2] \mid ax + by = \alpha\}  \]
\begin{equation}\label{hiper}
   H =\{(x,y)\in \rn[2] \mid ax + by = \alpha\}
\end{equation}
\end{lstlisting}

\tcblower

\[  
   H =\{(x,y)\in \rn[2] \mid ax + by = \alpha\}  
\]
\begin{equation}\label{hiper}
   H =\{(x,y)\in \rn[2] \mid ax + by = \alpha\}
\end{equation}
\end{tcolorbox}

O ambiente \ttt{eqnarray} enumera todas as suas equações.

\begin{tcolorbox}[title={Ambiente eqnarray}]
\begin{lstlisting}
\begin{eqnarray}
   H     & = & \{(x,y)\in \rn[2] \mid ax + by = \alpha\}\\ [5pt]
   d(x,y)& = & \sqrt[p]{\sum_{j=1}^{\infty}|x_j-y_j|^p}\\
   d(x,y)& \leq  & d(x,z) + d(z,y)
\end{eqnarray}
\end{lstlisting}

\tcblower
\begin{eqnarray}
   H     & = & \{(x,y)\in \rn[2] \mid ax + by = \alpha\}\\ [5pt]
   d(x,y)& = & \sqrt[p]{\sum_{j=1}^{\infty}|x_j-y_j|^p}\\ 
   d(x,y)& \leq  & d(x,z) + d(z,y)
\end{eqnarray}
\end{tcolorbox}

Se quiser retirar a numeração basta por um asterisco junto ao nome do ambiente, assim 
\verb|eqnarray*|. Se deseja retirar a numeração de apenas uma equação comoque o comando
\com{nonumber} na frente da equação que deseja retirar o número.

\subsection{Referência cruzada}\label{cruzada}

Chama-se referência cruzada aquela feita a uma parte da própria obra. Essa é uma 
demanda comum em matemática pois em muitas situações é necessário fazer referência 
a um resultado ou equação para justificar determinada conclusão, pode-se fazer 
referência a uma imagem ou gráfico, a uma função dentre outros elementos.

Em \LaTeX\ e na classe estilo,  todos os elementos que recebem uma numeração podem 
receber uma referência cruzada, inclui-se ai os teoremas, corolários, definições, 
lemas, imagens, tabelas, equações e  outros.


O mecanismo para fazer referência cruzada é composto por dois comandos
\begin{itemize}
 \item \cmc{label}{marca}: este comando marca o local/elemento a ser referenciado.
     A marca é um apelido escolhido livremente, a gosto do autor.
 \item \cmc{ref}{marca}: faz a referência ao objeto/elemento marcado com o comando 
     \cmc{label}{marca}. Para fazer referência a uma página utiliza-se o comando
     \cmc{pageref}{marca}
\end{itemize}

\begin{tcolorbox}[title={Exemplo de referência cruzada}]
   \begin{lstlisting}
   Na página~\pageref{prova}, a proposição~\ref{prova} estabele a 
   desigualdade triangular no conjunto $l^p$, ou seja,
   \[
    \sqrt[p]{\sum_{j=1}^{\infty}|x_j-y_j|^p}\leqslant
        \sqrt[p]{\sum_{j=1}^{\infty}\left(|x_j-z_j|\right)^p}+
        \sqrt[p]{\sum_{j=1}^{\infty}\left|z_j-y_j|\right)^p}
   \]
   Na página~\pageref{tab} a tabela~\ref{tab} mostra a primitiva de 
   três funções, na página~\pageref{hiper} a equação~\ref{hiper} 
   mostra que toda reta é um hiperplano em \rn[2] e, por fim, na 
   página~\pageref{Esquema} a imagem~\ref{Esquema} mostra a 
   hierarquia de várias estruturas algébricas importantes. 
   \end{lstlisting}
   
   \tcblower
   
   Na página~\pageref{prova}, a proposição~\ref{prova} estabele a desigualdade 
   triangular no conjunto $l^p$, ou seja,
   \[
    \sqrt[p]{\sum_{j=1}^{\infty}|x_j-y_j|^p}\leqslant
        \sqrt[p]{\sum_{j=1}^{\infty}\left(|x_j-z_j|\right)^p}+
        \sqrt[p]{\sum_{j=1}^{\infty}\left|z_j-y_j|\right)^p}
   \]
   Na página~\pageref{tab} a tabela~\ref{tab} mostra a primitiva de três 
   funções elementares, na página~\pageref{hiper} a equação~\ref{hiper} 
   mostra que toda reta é um hiperplano em \rn[2] e, por fim, na 
   página~\pageref{Esquema} a imagem~\ref{Esquema} mostra 
   a hierarquia de várias estruturas algébricas importantes.
\end{tcolorbox}

Observe que os códigos \verb|\pageref{prova}| e \verb|\ref{prova}| se transformaram 
nos números da página e da propposição respectivamente, isso é a referência cruzada.
O til antes desses comanndos impede a quebra de linha antes deles.

\section{Conectivos}

Com razoável frequência os conectivo "e", "ou"\,\, e "se"\,\, são
requisitados dentro de um ambiente matemático ou modo matemático,
locais em que os espaços são eliminados, para resolver esse problema e inserir
esses elementos com comodidade foram definidos os comandos
\begin{center}
	\com{e},\,\, \com{ou}\,\, e\,\, \com{se}
\end{center}

\begin{tcolorbox}
\begin{lstlisting}
$\nor{f} = 1  e f(x_0) = \nor{x_0}$ %%% Obeserve o efeito do "e"

$\nor{f} = 1 \e f(x_0) = \nor{x_0}$ %%% Obeserve o efeito do \e
\end{lstlisting}
\tcblower

$\nor{f} = 1  e f(x_0) = \nor{x_0}$ {\color{estilo}$\longleftarrow$ observe o efeito do "e".}

$\nor{f} = 1 \e f(x_0) = \nor{x_0}$ {\color{estilo}$\longleftarrow$ observe o efeito do \com{e}}
\end{tcolorbox}

\begin{tcolorbox}
\begin{lstlisting}
   \[ x\in A\cup B\Longrightarrow x\in A  ou x\in B \]	
   \[ x\in A\cup B\Longrightarrow x\in A \ou x\in B \]
\end{lstlisting}
	\tcblower
    \[ x\in A\cup B\Longrightarrow x\in A  ou x\in B \]	
    \[ x\in A\cup B\Longrightarrow x\in A \ou x\in B \]
\end{tcolorbox}

O comando \com{se} funciona da mesma forma.
\begin{tcolorbox}
\begin{lstlisting}
\[  |a| = \left\{\begin{array}{r}
                a, \se a>0\\
                0, \se a=0\\
               -a, \se a<0
             \end{array}
         \right.
\]
\end{lstlisting}
\tcblower
\[
|a| = \left\{\begin{array}{r}
                a, \se a>0\\
                0, \se a=0\\
               -a, \se a<0
             \end{array}
\right.
\]
\end{tcolorbox} 