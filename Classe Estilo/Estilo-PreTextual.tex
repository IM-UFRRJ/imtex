
\chapter{Parte pré-textual}

Os comandos do capítulo~\ref{identifica} devem ficar no preâmbulo do arquivo
\ttt{.tex}, exceto os relativos a data. Eles são usados pelos
comandos e ambientes responsáveis pela confecção dos elementos
da parte pré-textual, os quais seguem na mesma ordem que devem ficar no texto.

\section{Capa}

Prestadas as informação descritas no capítulo~\ref{identifica} basta
inserir \com{capa} onde a capa deve ser criada, simples assim.

Os comandos  \com{fontecapa} e \com{fontecapatitle} definem a fonte da capa e
do título do trabalho na capa respectivamente. Ambos podem ser alterados com o
\coma{reewcommand}.

\section{Página de rosto}

O mecanismo para criar página de rosto e capa é o mesmo, só
muda o nome do comando, que neste caso é \com{rosto}. Como a
página de rosto é a segunda página este comando deve vir depois
do \com{capa}. Tem-se
\begin{description}
  \item[\com{Trosto}:] define a largura do texto na página de
      rosto. Se valor padrão é $65\%$ da largura do texto. Esse
      comando é um length e deve ser alterado da seguinte forma
      \verb|\setlength{\Trosto}{Novo valor}|;
  \item[\com{Lrosto}:] determina o comprimento da linha sobre
      o nome do orientador e coorientador na página de rosto. Seu
      valor padrão é \comando{Trosto}, ou seja, a largura do texto
      na página de rosto. É um length e portanto deve ser alterado
      com o comando \comando{setlength}. Por exemplo
      \verb|\setlength{\Lrosto}{Novo valor}|;
  \item[\com{Erosto}:] Espessura da linha sobre o nome do
      orientador/coorientador na página de rosto. Seu valor padrão
      é $0.7$pt. Também é um length e pode ser alterado com
      \verb|\setlength{\Erosto}{Novo valor}|.
\end{description}


Os comandos \com{capa} e \com{rosto} oferecem \ttt{capa} e \ttt{folha de rosto} segundo as normas do curso de Matemática. Mas podem ser feitos outros modelos que também atendam essas regras.

\section{Epígrafe}

O comando \com{epigrafe} define uma epígrafe e seu autor.
\begin{tcolorbox}
O fragmento
   \begin{lstlisting}
      \epigrafe{Para os pais, filho sempre é inteligente, bonito e santo.}{Benaia Sobreira}
   \end{lstlisting}
\tcblower
depois de compilado produz
   \begin{flushright}
 	  {\footnotesize Para os pais, filho sempre é inteligente, bonito e santo.\hspace{0.6cm}\emph{Benaia Sobreira}}
   \end{flushright}
%%% Não era conveniente usar o \epigrafe aqui por que ele limpa o cabeçalho.
%%% Mas seu efeito foi reproduzido fielmente.
\end{tcolorbox}

\section{Dedicatória}

A dedicatória é opcional, sua confecção não está sujeita a normas
estritas, assim pode ser construída livremente, mas, tenha bom senso.

A classe \estilo{} oferece o ambiente \ttt{dedico} para
construir a dedicatória. O título é centralizado e definido
pelo comando \comando{dedi}, com fonte \verb|\caligrafica| em
tamanho \verb|\Huge|. O \comando{dedi} pode ser redefinido com
\comando{renewcommand}.
\begin{tcolorbox}
\begin{lstlisting}
\begin{dedico}
   Dedico esse trabalho a todos os alunos do curso de
   Matemática que gostam e se empenham em fazer trabalhos bonitos.
\end{dedico}
\end{lstlisting}
\end{tcolorbox}

\section{Agradecimento}

Análogo a dedicatória, substitua o ambiente \ttt{dedico} por
\ttt{agradece} e o comando \coma{dedi} por \comando{agrada}.
\begin{tcolorbox}
\begin{lstlisting}
\begin{agradece}
   A Deus, principio e fim. A meu alunos, colaboradores dedicados,
   pacientes e divertidos. Em particular agradeço as alunas da
   primeira turma: Camila, Rachel, Ursula, Silvia, Rayane, $\ldots$.
   E aos alunos: Rubens, Bruno, A Deus, princípio e fim.
\end{agradece}
\end{lstlisting}
\end{tcolorbox}

\section{Resumo em Língua Vernácula e Estrangeira}\label{resumo}
O resumo em língua vernácula e estrangeira é obrigatório.
A classe \estilo{} oferece os ambientes \ttt{resumo} e
\ttt{resumoE}, respectivamente, para construí-los conforme
normatização vigente.

Uma lista de palavras-chave do texto deve ser incorporada ao resumo.
Para a língua vernácula existe o comando \com{palavras} e para
a língua estrangeira, qualquer que seja ela, \com{palavrasE}.
\begin{tcolorbox}
\begin{lstlisting}
\begin{resumo}
   A classe \estilo{} foi construída sobre a classe base report,
   todas as opções desta foram preservadas. Suas opções padrão
   atendem as normas do Departamento de Tecnologias e Linguagens - DTL
   para confecção de trabalhos acadêmicos/científicos. Porém, é uma
   classe muito flexível.

   \palavras{\TeX, \LaTeX, Padronização, Beleza, Tipografia,
   Coesão e Simplicidade}
\end{resumo}
\end{lstlisting}
\end{tcolorbox}

\section{Sumário}

Insira o comando \com{tableofcontents} onde o sumário
deve ser criado. Pronto! Está criado um sumário coerente, correto,
bonito e que atende às normas da ABNT e internacionais.

O \com{tableofcontents} compõe o sumário utilizando os títulos
de capítulos, seções e subseções definidos pelos comandos \comando{chapter},
\comando{section} e \comando{subsection} respectivamente. O
seccionamento gerado com: \comando{chapter*}, \comando{section*}
e \comando{subsection*} não tem seu título incluído no sumário.
Para incluir um título avulso no nível dos capítulos pode-se usar
\begin{tcolorbox}
\begin{lstlisting}
\addcontentsline{toc}{chapter}{título a ser inserido no sumário}
\end{lstlisting}
\end{tcolorbox}

O comando \com{tableofcontents} que cria o sumário não sofreu qualquer
alteração, por isso todos os seus elementos podem ser controlados com
as ferramentas do koma-script. Por exemplo, a fonte do título dos
capítulos no sumário foi estabelecida com a definição
\begin{center}
 \cmc{addtokomafont}{chapterentry}$\{$\verb|\normalfont\scshape\color{Navy}|$\}$
\end{center}

\section{Lista de figuras}

O comando \com{listoffigures} cria uma lista de figuras, um
sumário das figuras do documento criadas dentro de um ambiente
figure. Nela, cada entrada corresponde a uma legenda definida
pelo comando \com{caption}. No capítulo~\ref{imagens} existem 
vários exemplo de como inserir imagem. Figuras sem legendas não são
consideradas.

As entradas na lista de figuras respeitam a ordem de aparição do
\ttt{figure} no documento.  Lista de figuras e sumário recebem a 
mesma formatação.

\section{Lista de tabelas}

O comando \com{listoftable} cria uma lista de tabelas, um
sumário das tabelas do documento criadas dentro de um
ambiente table. Nela, cada entrada corresponde a uma legenda
definida pelo comando \com{caption}. Tabelas sem legendas
não são consideradas.

As entradas na lista de tabelas respeitam a ordem de aparição do
\ttt{table} no documento. Lista de tabelas e sumário recebem a 
mesma formatação.

\begin{tcolorbox}[title={Exemplo de inserção de tabela}]
\begin{table}[H]
  \centering
  \begin{tabular}{|c|c|c|}
    \hline
    $f(x)$   & $f'(x)$        &$\displaystyle\int f(x)\,dx$ \\ \hline
    $x$      & $1$            &$\dfrac{x^{2}}{2} + c$ \\
    $e^{x}$  & $e^{x}$        &$e^{x} + c$    \\
    $\ln(x)$ & $\frac{1}{x}$ &$x\ln(x) - x$  \\ \hline
  \end{tabular}
  \caption{Integrais elementares}\label{tab}
\end{table}
  \[
  \begin{array}{|c|c|c|}
    \hline
    f(x)   &  f'(x)       & \displaystyle\int f(x)\,dx \\[5pt] \hline
    x      &  1           & \dfrac{x^{2}}{2} + c\\ [7pt] \hline
    e^{x}  &  e^{x}       & e^{x} + c \\ \hline
    \ln(x) & \frac{1}{x} & x\ln(x) - x \\ \hline
  \end{array}
  \]
\tcblower
\begin{lstlisting}
\begin{table}[H]
  \centering
  \begin{tabular}{|c|c|c|}
    \hline
    $f(x)$   & $f'(x)$        &$\displaystyle\int f(x)\,dx$ \\ \hline
    $x$      & $1$            &$\dfrac{x^{2}}{2} + c$ \\
    $e^{x}$  & $e^{x}$        &$e^{x} + c$    \\
    $\ln(x)$ & $\frac{1}{x}$ &$x\ln(x) - x$  \\ \hline
  \end{tabular}
  \caption{Integrais elementares}\label{tab}
\end{table}
  \[
  \begin{array}{|c|c|c|}
    \hline
    f(x)   &  f'(x)       & \displaystyle\int f(x)\,dx \\[5pt] \hline
    x      &  1           & \dfrac{x^{2}}{2} + c\\ [7pt] \hline
    e^{x}  &  e^{x}       & e^{x} + c \\ \hline
    \ln(x) & \frac{1}{x} & x\ln(x) - x \\ \hline
  \end{array}
  \]
\end{lstlisting}
\end{tcolorbox}

Os comandos \verb|[5pt]| e \verb|[7pt]| aumentaram em 
5pt e 7pt o espaço entre linhas. 

Como a primeira tabela está dentro de um ambiente table e tem uma legenda
criada com o comando \com{caption}, sua legenda será uma entrada na lista
de tabelas. A segunda tabela não tem legenda, e por isso não gera entrada 
na lista de tabelas.

Para criar uma tabela com legenda mas que não gere entrada na lista de 
tabelas use o comando \com{caption*}.

A primeira tabela foi criada com o ambiente tabular, como seus elementos são 
entes matemáticos foi preciso colocá-los entre dolares. A segunda tabela foi
criada com o ambiente array que aceita tanto o uso de dólares quanto aceita 
ser posto dentro de um ambiente matemático. Veja no exemplo que a utilização 
do ambiente \verb|\[  \]| dispensou o uso dos muitos dólares empregados na 
construção da primeira tabela.

\begin{table}[H]
  \centering
  \rowcolors{1}{estilo!20}{estilo!15} %%% Insere a cor nas linhas
  \begin{tabular}{ccc}
    $f(x)$   & $f'(x)$        &$\displaystyle\int f(x)\,dx$ \\ 
    $x$      & $1$            &$\dfrac{x^{2}}{2} + c$ \\
    $e^{x}$  & $e^{x}$        &$e^{x} + c$    \\
    $\ln(x)$ & $\frac{1}{x}$  &$x\ln(x) - x$   
  \end{tabular}
  \caption*{Essa legenda não vai para a lista de tabelas}
\end{table}
  \[
  \rowcolors{1}{Navy!20}{Navy!15} %%% Insere a cor nas linhas
  \begin{array}{ccc}
    f(x)   &  f'(x)       & \displaystyle\int f(x)\,dx \\[5pt] 
    x      &  1           & \dfrac{x^{2}}{2} + c\\ [7pt] 
    e^{x}  &  e^{x}       & e^{x} + c \\ 
    \ln(x) & \frac{1}{x} & x\ln(x) - x 
  \end{array}
  \]
\begin{lstlisting}
\begin{table}[H]
  \centering
  \rowcolors{1}{estilo!20}{estilo!15} %%% Insere a cor nas linhas
  \begin{tabular}{ccc}
    $f(x)$   & $f'(x)$        &$\displaystyle\int f(x)\,dx$ \\ 
    $x$      & $1$            &$\dfrac{x^{2}}{2} + c$ \\
    $e^{x}$  & $e^{x}$        &$e^{x} + c$    \\
    $\ln(x)$ & $\frac{1}{x}$  &$x\ln(x) - x$   
  \end{tabular}
  \caption*{Essa legenda não vai para alista de tabelas}
\end{table}
  \[
  \rowcolors{1}{Navy!20}{Navy!15} %%% Insere a cor nas linhas
  \begin{array}{ccc}
    f(x)   &  f'(x)       & \displaystyle\int f(x)\,dx \\[5pt] 
    x      &  1           & \dfrac{x^{2}}{2} + c\\ [7pt] 
    e^{x}  &  e^{x}       & e^{x} + c \\ 
    \ln(x) & \frac{1}{x} & x\ln(x) - x 
  \end{array}
  \]
\end{lstlisting}




